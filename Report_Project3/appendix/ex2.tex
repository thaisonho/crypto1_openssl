\section{Bài 2: Cách thức chạy chương trình và Video Demo}
\label{apd:bai2_tutorial}

Chương trình mã hóa và giải mã RSA được viết bằng ngôn ngữ lập trình Python 3.x với thư viện \texttt{cryptography}. Mã nguồn nằm trong thư mục \texttt{Source/Bai\_2}.

\subsection{Yêu cầu hệ thống}
\begin{itemize}
    \item Python 3.8 trở lên
    \item OpenSSL (để tạo khóa và kiểm tra tương thích)
    \item Thư viện \texttt{cryptography}
\end{itemize}

\subsection{Cài đặt môi trường}

Để thiết lập môi trường và cài đặt các dependency cần thiết:

\begin{lstlisting}[language=bash]
# Tao moi truong ao (neu chua co)
python -m venv .venv

# Kich hoat moi truong ao
source .venv/bin/activate  # cho Linux/Mac
# hoac
.venv\Scripts\activate     # cho Windows

# Cai dat thu vien cryptography
pip install cryptography
\end{lstlisting}

\subsection{Cấu trúc thư mục Bài 2}

\begin{lstlisting}[language=bash]
Source/Bai_2/
    rsa_encrypt.py    # Chuong trinh ma hoa RSA
    rsa_decrypt.py    # Chuong trinh giai ma RSA
    priv.pem          # Khoa bi mat RSA (2048-bit)
    pub.pem           # Khoa cong khai RSA
    plain             # File ban ro mau
    cipher            # File ban ma (da duoc ma hoa)
    README.md         # Huong dan su dung
\end{lstlisting}

\subsection{Hướng dẫn sử dụng}

\subsubsection{Bước 1: Tạo khóa RSA bằng OpenSSL}

\begin{lstlisting}[language=bash]
# Di chuyen den thu muc Bai_2
cd Source/Bai_2

# Tao khoa bi mat RSA 2048-bit
openssl genpkey -out priv.pem -algorithm RSA -pkeyopt rsa_keygen_bits:2048

# Tao khoa cong khai tu khoa bi mat
openssl pkey -in priv.pem -out pub.pem -pubout
\end{lstlisting}

\subsubsection{Bước 2: Tạo file bản rõ}

\begin{lstlisting}[language=bash]
echo "Hello! Day la tin nhan bi mat can ma hoa RSA." > plain
\end{lstlisting}

\subsubsection{Bước 3: Mã hóa bằng chương trình Python}

\begin{lstlisting}[language=bash]
python rsa_encrypt.py <pub.pem> <plain> <cipher>
\end{lstlisting}

Trong đó:
\begin{itemize}
    \item \texttt{<pub.pem>}: Tệp chứa khóa công khai RSA
    \item \texttt{<plain>}: Tệp chứa bản rõ cần mã hóa
    \item \texttt{<cipher>}: Tệp đầu ra chứa bản mã
\end{itemize}

Ví dụ:
\begin{lstlisting}[language=bash]
python rsa_encrypt.py pub.pem plain cipher_output
\end{lstlisting}

\subsubsection{Bước 4: Giải mã bằng chương trình Python}

\begin{lstlisting}[language=bash]
python rsa_decrypt.py <priv.pem> <cipher> <plain>
\end{lstlisting}

Trong đó:
\begin{itemize}
    \item \texttt{<priv.pem>}: Tệp chứa khóa bí mật RSA
    \item \texttt{<cipher>}: Tệp chứa bản mã cần giải mã
    \item \texttt{<plain>}: Tệp đầu ra chứa bản rõ
\end{itemize}

Ví dụ:
\begin{lstlisting}[language=bash]
python rsa_decrypt.py priv.pem cipher_output plain_decrypted
\end{lstlisting}

\subsection{Kiểm tra tính tương thích với OpenSSL}

\subsubsection{Kịch bản 1: Mã hóa bằng Python, giải mã bằng OpenSSL}

\begin{lstlisting}[language=bash]
# Ma hoa bang Python
python rsa_encrypt.py pub.pem plain cipher_py

# Giai ma bang OpenSSL
openssl pkeyutl -in cipher_py -out plain_check -inkey priv.pem -decrypt

# So sanh ket qua
cat plain_check
diff plain plain_check
\end{lstlisting}

\subsubsection{Kịch bản 2: Mã hóa bằng OpenSSL, giải mã bằng Python}

\begin{lstlisting}[language=bash]
# Ma hoa bang OpenSSL
openssl pkeyutl -in plain -out cipher_openssl -inkey pub.pem -pubin -encrypt

# Giai ma bang Python
python rsa_decrypt.py priv.pem cipher_openssl plain_check

# So sanh ket qua
cat plain_check
diff plain plain_check
\end{lstlisting}

\subsection{Video Demo}

Video Demo Bài tập 2 có thể được xem tại đây:

\url{https://drive.google.com/file/d/YOUR_VIDEO_ID/view?usp=sharing}

\textit{(Lưu ý: Thay thế link trên bằng link video thực tế sau khi upload)}
