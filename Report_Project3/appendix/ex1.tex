\appendix
\section{Bài 1: Cách thức chạy chương trình và Video Demo}
\label{apd:tutorial}
Chương trình được viết bằng ngôn ngữ lập trình Python 3.13. Vì vậy để chạy chương trình, trước hết máy tính phải cài đặt Python.

Tham khảo cài đặt Python tại đây: \url{https://www.python.org/downloads/release/python-3135/}.

Để thiết lập chạy chương trình cần thực hiện các bước sau:

\begin{lstlisting}[language=bash]
python -m venv .venv
source .venv/bin/activate  # cho Linux/Mac user
# hoac
.venv\Scripts\activate # cho Windows
pip install -r requirements.txt
\end{lstlisting}

Đây là các câu lệnh để thiết lập môi trường ảo và cài đặt dependency cho chương trình.

Sau đó chạy chương trình với lệnh
\begin{lstlisting}[language=bash]
# Sinh private key voi OpenSSL
openssl genpkey -algorithm rsa -pkeyopt rsa_keygen_bits:<NUMBITS> -out <private.pem>
# Sinh public key voi private key vua tao
openssl pkey -in <private.pem> -out <public.pem> -pubout
# Voi bai tap 1
python rsa_key_parser.py <private.pem> <public.pem>
# Optional: So sanh output cua chuong trinh voi output cua OpenSSL
openssl rsa -check -in <private.pem> -noout -text
\end{lstlisting}

Trong đó:
\begin{itemize}
    \item \texttt{<NUMBITS>} là số lượng bit mong muốn.
    \item \texttt{<private.pem>} Là tập tin private key tạo bằng OpenSSL.
    \item \texttt{<public.pem>} Là tập tin public key tạo bằng OpenSSL với file private key trên.
\end{itemize}

Video Demo Bài tập 1 có thể được xem tại đây: 

\url{https://drive.google.com/file/d/1li74oNObiWzrpEIMl30stQKqJfKSmP9F/view?usp=sharing}